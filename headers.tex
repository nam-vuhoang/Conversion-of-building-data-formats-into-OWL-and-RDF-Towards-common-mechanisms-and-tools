\usepackage[numbers]{natbib}% for bibliography sorting/compressing
%\usepackage{amsmath}
%\usepackage{endnotes}

\usepackage{graphicx} % needed for command \includegraphics, also supports width

%%%%%%%%%%% Put your definitions here


%%%%%%%%%%%%%%%%%%%%%%%%%%%%%
% User-defined commands
%%%%%%%%%%%%%%%%%%%%%%%%%%%%%


\usepackage{tabulary}
\usepackage{multirow}
% \usepackage{booktabs} % provides \toprule, \midrule, \bottomrule for table


% \usepackage{geometry}
\usepackage{color}
\usepackage{colortbl} % needed for coloring table, see: https://texblog.org/2011/04/19/highlight-table-rowscolumns-with-color/
\usepackage[table]{xcolor} % needed for coloring table, see: https://en.wikibooks.org/wiki/LaTeX/Tables

%\definecolor{name}{system}{definition}
\definecolor{Gray}{gray}{0.9}
\definecolor{LightCyan}{rgb}{0.88,1,1}




\setlength{\paperheight}{297mm} % needed for package {hyperref}

%\usepackage{hyperref}
\usepackage[draft]{hyperref} % defined as 'draft' to avoid error message http://tex.stackexchange.com/questions/1522/pdfendlink-ended-up-in-different-nesting-level-than-pdfstartlink

\usepackage{url}


%
% TABULARS
%

% This command is to add new lines in tables
% See manual: http://tex.stackexchange.com/questions/2441/how-to-add-a-forced-line-break-inside-a-table-cell

\newcommand{\specialcell}[2][c]{%
  \begin{tabular}[#1]{@{}l@{}}#2\end{tabular}}



%
% LISTINGS
%

\usepackage{listings} % see manual: ftp://ftp.funet.fi/pub/TeX/CTAN/macros/latex/contrib/listings/listings.pdf

%
% Change caption font size: http://tex.stackexchange.com/questions/129291/change-caption-style-of-listings-package-without-caption-package
%
\usepackage[font=bf,skip=\baselineskip]{caption}
\captionsetup[lstlisting]{font={footnotesize,sf}}

% language Ruby is used to enable comment lines started with hash character

\lstset{language=Ruby,
aboveskip=6pt,
belowskip=6pt,
backgroundcolor=\color[gray]{0.85},
numberstyle=\scriptsize,
basicstyle=\linespread{0.9}\ttfamily\scriptsize,
% aboveskip={-4ex},
numbers=none,
stepnumber=1,
frame=none,
breaklines=true,
breakautoindent=true}

\AtBeginDocument{%
  \renewcommand{\thelstlisting}{\arabic{lstlisting}}
 }

%\spnewtheorem{principle}{Principle}{\bfseries}{\rmfamily}

% \AtBeginDocument{
%     \newtheoremstyle{principleStyle}
%         {1.5\topsep} % Space above, default \topsep
%         {0.5\topsep} % Space below, default \topsep
%         {\itshape} % Body font
%         {} % Indent amount, example: \parindent
%         {\bfseries\itshape} % Theorem head font
%         {.} % Punctuation after theorem head
%         {.5em} % Space after theorem head
%         {} % Theorem head spec (can be left empty, meaning `normal')
        
%     \theoremstyle{principleStyle} \newtheorem{principle}{Principle}
% }




% Commands for C#, C++
% Source: https://www.johndcook.com/blog/2011/10/18/typesetting-c-in-latex/
\newcommand{\CPP}{C\nolinebreak\hspace{-.05em}\raisebox{.4ex}{\tiny\bf +}\nolinebreak\hspace{-.10em}\raisebox{.4ex}{\tiny\bf +}}
\newcommand{\CS}{C\nolinebreak\hspace{-.05em}\raisebox{.6ex}{\scriptsize\bf \#}}






%\usepackage{linegoal}

% \newcommand{\definition}[1]{\textbf{#1}}
%\newcommand{\name}[1]{\emph{#1}}

% programming language names
\newcommand{\name}[1]{\textsf{\footnotesize{}#1}}

% highlight texts
%\newcommand{\highlight}[1]{\colorbox{yellow}{\parbox[t]{\linegoal}{#1}}}
\newcommand{\highlight}[1]{\colorbox{yellow}{\parbox[t]{\hsize}{#1}}}


% TODO text
% \newcommand{\TODO}[1]{\highlight{\textbf{TODO: }{#1}}}
\newcommand{\TODO}[1]{{}}

%\newcommand{\REVIEW}[1]{\textit{#1}}

% ifcOWL layer names
\newcommand{\bname}[1]{\textnormal{\textbf{#1}}}

\newcommand{\simple}{\bname{Simple}}
\newcommand{\standard}{\bname{Standard}}
\newcommand{\extended}{\bname{Extended}}

\newcommand{\ifcowl}{\bname{ifc\-OWL}}
\newcommand{\ifcrdf}{\bname{ifc\-RDF}}
\newcommand{\ifcsimple}{\ifcowl{}-\simple{}}
\newcommand{\ifcstandard}{\ifcowl{}-\standard{}}
\newcommand{\ifcextended}{\ifcowl{}-\extended{}}

\newcommand{\ifcconverter}{\bname{IFC2LD}}
\newcommand{\ifcld}{\ifcconverter}


%%%%%%%%%%% End of definitions

\pubyear{2017}
\volume{0}
\firstpage{1}
\lastpage{1}