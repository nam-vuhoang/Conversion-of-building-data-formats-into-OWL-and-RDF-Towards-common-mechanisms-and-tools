\subsection{Glossary and acronyms}\label{sec:glossary}

In the Information and Communication Technology (ICT) and the AEC/FM industries, many terms are overlapped, ambiguous and may confuse readers, especially if they are unfamiliar with the specific data level terms of Building Information Modeling (BIM).
Firstly, both the software engineering and construction engineering life cycles include similar stages, namely analysis, design, implementation, testing and maintenance.
Construction engineers by default understand term \emph{de\-sign mod\-el} as a data model related to their (AEC) design discipline, whereas software engineers may interpret it in a different way.
Secondly, there are many specific terms used in BIM.
Readers may not know, for instance, that term \emph{prod\-uct da\-ta mod\-el} denotes an information model for product data, where \emph{prod\-uct} is a construction result like a building or an infra structure; \emph{prod\-uct mod\-el} is an instantiation of a \emph{prod\-uct da\-ta mod\-el}, i.e. a model of a specific product; and \emph{in\-fra\-struc\-ture prod\-uct mod\-el}, \emph{build\-ing in\-for\-ma\-tion mod\-el} and \emph{IFC object mod\-el} are particular cases of \emph{prod\-uct mod\-el}.
As can be noted, \emph{building information} means information about building, not an action to build information; \emph{prod\-uct da\-ta mod\-ell} designates an information model for product data, not a data model produced.
Thirdly, there are many words which are often considered as synonyms but they are not, for example, \emph{da\-ta sche\-ma} and \emph{da\-ta mod\-el}, or \emph{in\-fra\-struc\-ture in\-for\-ma\-tion model} and \emph{in\-fra\-struc\-ture da\-ta mod\-el}.
Lastly, one acronym may denote different meanings depending on the context.
For instance, in literature concerning BIM application, the abbreviation IFC (Industry Foundation Classes) most often means the IFC data format.
However, it may also indicate the IFC data schema, the IFC conceptual data model, the IFC standard specification, or the IFC platform.


% The following glossary (see Table \ref{table:glossary}) is intended to assist readers 


The following glossary is intended to assist readers in explicit understanding terms used in this paper.
Most of these concepts are cited and summarised from InfraBIM Glossary written by Kalle Ser\'en for buildingSMART Finland \cite{seren2014infrabim}.
Part 1 consists of generic terms.

\begin{itemize}
    \item \emph{In\-for\-ma\-tion mod\-el} or \emph{Con\-cep\-tu\-al mod\-el} - Formal definition of information, which defines the elements of information and their relationships.
    \item \emph{Da\-ta mod\-el} - Usually used as a synonym for the terms conceptual or information model.Note: Sometimes a distinction is made between the implementation technology independent conceptual model, and thedata model which is planned for a specific implementation technology (e.g. the data structures of a database).Note: An Information model for Product data is called a Product data model.In IFCs, the IFC product data model (or project data model) is called IFC Object model.
    
    
    \item \emph{Prod\-uct da\-ta} or \emph{Prod\-uct mod\-el da\-ta} - A representation of information about a product in a formal manner suitable for communication, interpretation, or processing by human beings or by computer applications.
    \item \emph{Prod\-uct da\-ta mod\-el} or \emph{Prod\-uct in\-for\-ma\-tion mod\-el} - An information model for product data.
    \item \emph{Prod\-uct mod\-el} - An instantiation of a product data model. For example, a model of a specific infrastructure stored into an exchange file in the LandXMLformat.
    \item \emph{Prod\-uct mod\-el} - An instantiation of a product data model. For example, a model of a specific infrastructure stored into an exchange file in the LandXMLformat.
    \item \emph{XXX prod\-uct da\-ta mod\-el} - Particular cases of \emph{Prod\-uct da\-ta mod\-el}. XXX is a product type such as  \emph{build\-ing} or \emph{in\-fra\-struc\-ture}.
    \item \emph{XXX prod\-uct mod\-el} or \emph{XXX information model} - Particular cases of \emph{Prod\-uct mod\-el}.
    \item \emph{XXX product mod\-el\-ling} or \emph{XXX information modeling} - An art and science that deals with modelling and representation and exchange of XXX-related information in computer interpretable form.
    % \item \emph{InfraBIM} - Acronym for Infra Built Environment Information Modelling. 
    \item \emph{Data sharing} - Common access to data in a database by a number of applications that create, use and update the data.
    \item \emph{Data exchange} - The exchange of data between computer applications; typically using an data exchange file. 
    \item \emph{Data exchange format} - A computer interpretable format used for storing, accessing, transferring, and archiving data.
    \item \emph{Native model} - A model stored in a specific application program's storage format.
    \item \emph{As-planned model}
    \item \emph{As-built model}
    \item \emph{Maintenance model}
    \item \emph{Model-based} - An approach, e.g. for AEC/FM computer applications, where the target system (e.g. building) is represented by a model which then is used as a basis for analysis, creation of presentations, reports, and exchange of data.
    \item \emph{Document-based} - A paradigm in which information is represented using text documents, drawings, etc., which require interpretation by human.
    \item \emph{Product structure} - Describes the decomposition of a product from its components.
    \item \emph{Information management}.

    
    \item \emph{Mapping} - The process of converting a data set from one form to another form. The forms may for example be defined by different schemas.
    \item \emph{Metadata} - Data about data.
    \item \emph{Schema} - A model that defines representation of information. In EPXRESS, a schema is a unit or module within the whole EXPRESS model that addresses a specific subdomain of the model. Schemas of an EXPRESS model may be related to each other via schema interfaces.
    
    \item \emph{Exchange format} - The  syntax for representing data for exchange purposes.

    
    \item \emph{Formal} - Described in an unambiguous, systematic way.
    \item \emph{EXPRESS} - 
    \item \emph{Concept}
    \item \emph{Data type} 
    \item \emph{Class} vs. \emph{Entity type}
    \item \emph{Class hierarchy} 
    \item \emph{Inheritance} 
    \item \emph{Inheritance hierarchy} 
    \item \emph{Object} vs. \emph{Entity}
    \item \emph{Cardinality} 
    \item \emph{Attribute} 
    \item \emph{Complex instance}
    \item \emph{Constructed Data types}
    \item \emph{Explicit attribute} 
    \item \emph{Derived attribute} 
    \item \emph{Inverse attribute} 
    \item \emph{Entity type} 
    \item \emph{Composition} or \emph{Aggregation}
    \item \emph{Inverse relationship}
    \item \emph{Extension}
    \item \emph{Full model exchange}
    \item \emph{Partial model exchange}
    \item \emph{Object-oriented}
    \item \emph{Optional attribute}
    \item \emph{Rule} 
    
    
    \item \emph{Industry Foundation Classes} - 
    \item \emph{ifcXML} - 
    \item \emph{IFC Object Model} or \emph{IFC-model} or \emph{IFC product data model} - 
    \item \emph{IFC data schema} - 
    \item \emph{IFC Platform} - 
    \item \emph{IFC specification} - 
    \item \emph{IFC Toolit} - 

    
    
    % \item \emph{data schema}
    % \item \emph{data model}
    % \item \emph{(data) conceptual model}
    % \item \emph{data format}
    % \item \emph{data serialisation format}
    % \item \emph{ontology}
    % \item \emph{dataset}
    % \item \emph{data type (or type)}
    % \item \emph{entity type (or class)}
    % \item \emph{"pure" data}
    % \item \emph{static data}
\end{itemize}



Besides, to distinguish schemas of the same data model but in different languages, they are denoted in form \texttt{<da\-ta\-Mod\-el>-<sche\-ma\-Lan\-guage>}, for instance, IFC-EXPRESS, IFC-XSD, and IFC-OWL.
Similarly, datasets of the same data model but in different serialization formats are designated as \texttt{<da\-ta\-Mod\-el>-<da\-ta\-For\-mat>}, for example, IFC-SPF, IFC-XML, IFC-ZIP.


Part 2 consists of acronyms which are specially defined in this paper to avoid ambiguity.

\begin{itemize}
    \item \emph{BIM} - Building Information Modelling. 
    \item \emph{IFC-EXPRESS} 
    \item \emph{IFC-SPF} 
    \item \emph{IFC-XSD} or \emph{ifcXML}
    \item \emph{IFC-XML} 
    \item \emph{IFC-OWL} or \emph{ifcOWL}
    \item \emph{IFC-RDF} or \emph{ifcRDF}
    \item \emph{IFC-EXPRESS-to-OWL}
    \item \emph{IFC-SPF-to-RDF}
    \item \emph{IFC-to-LD}
\end{itemize}


