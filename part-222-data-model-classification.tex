\subsubsection{Data model classification}\label{sec:data-model-classification}

Our survey shows that industrial data models specified with the same schema language have many similarities, such as data types, data structures, and object patterns being used within them.
Therefore, in order to facilitate data model analysis, all data models are divided into three groups, namely (1) EXPRESS-based data models, (2) XSD-based data models, and (3) Tabular and relational data models (see \autoref{table:data-model-classification}).


\begin{table}
    \centering
    \caption{Data model classification}
    \begin{tabularx}{\columnwidth}{llll}
        % \hline
        %     Data model &
        %     \specialcell[t]{EXPRESS-\\based} &
        %     \specialcell[t]{XSD-\\based} &
        %     \specialcell[t]{Tabular \&\\relational}
        % \\
        \hline
            Data models &
            EXPRESS- &
            XSD- &
            Tabular \&
        \\
             &
            based &
            based &
            relational
        \\
        \hline
            Schema languagse &
            EXPRESS &
            XSD &
            \specialcell[t]{CSV,\\Spreadsheet,\\SQL}
        \\            
        \hline
            Data formats &
            \specialcell[t]{STEP-File\\(SPF),\\STEP-XML} &
            XML &
            \specialcell[t]{CSV,\\Spreadsheet,\\RDBMS}
        \\            
        \hline
            Examples &
            \specialcell[t]{IFC,\\COBie} &
            \specialcell[t]{IFC,\\COBie,\\bsDD,\\MVD,\\CityGML,\\InfraXML,\\LandXML,\\private\\data\\formats} &
            \specialcell[t]{COBie,\\private\\data\\formats}
        \\
        \hline
    \end{tabularx}
    \label{table:data-model-classification}
\end{table}


% \begin{tabularx}{\textwidth}{|X|X|X|}
%   \hline
%   foo   & bar    & fubar \\
%   fubar & toobar & foo \\
%   \hline
% \end{tabularx}


\textbf{1. EX\-PRESS-ba\-sed da\-ta mod\-els} have data schemas specified in EXPRESS language \cite{iso2004express} and their datasets are stored in STEP-File or STEP-XML formats.
EXPRESS is a standard data modeling language for product data that is defined in ISO 10303-11 \cite{iso2004express}, a part of ISO 10303 Product data representation and exchange standard, as known as ISO STEP (Standard for the Exchange of Product model data).
ISO STEP also specifies EXPRESS-G, a graphical notation for a subset of EXPRESS.
An EXPRESS-based data model may have more than one schemas, each of which address a specific subdomain of the domain.
These schemas may be related to each other via schema interfaces.
STEP-File (STEP Physical File, SPF, or STEP) \cite{iso2016stepfile} and STEP-XML \cite{iso2007stepxml} are two data serialization formats, accordingly defined in ISO 10303-21 and ISO 10303-28.




\textbf{2. XSD-\-ba\-sed da\-ta mod\-els} have data schemas specified in XSD (W3C XML Schema Definition) language and their datasets are stored in XML format.
XSD is one of several XML schema languages; its latest version XSD 1.1 is published as a World Wide Web Consortium (W3C) Recommendation in 2012 \cite{gao2009w3c, peterson2009w3c}.



\textbf{3. Tabular and relational data models} group have three subgroups.







% \begin{enumerate}

%     \item[G1] \emph{EXPRESS-based data models} contain data models defined in the data specification language EXPRESS. Their datasets are serialised 
%     \item[G2] \emph{XSD-based data models}
%     \item[G3] \emph{Tabular and relational data models}

% \end{enumerate}

% All three levels of each data model are analysed with regards to the conversion to OWL ontologies and RDF datasets: the data schema definition language, the data schema, and the dataset.










% Below are several typical examples, which are official or de-facto standards:









% \textbf{Model View Definition XML (mvdXML)}...









% \textbf{City Geography Markup Language (CityGML)}...





% The scope of the research interest includes a small, but central part among plenty of structured data models used in the AEC/FM industry.

% Currently, the AEC/FM industry plenty of structured data models, which have schemas specified in different definition languages and which can be serialised in multiple formats.



    






% \subsubsection{Typical examples}\label{building-data-formats-examples}


% Currently, the AEC/FM industry uses plenty of structured data models, which have schemas specified in different definition languages and which can be serialised in multiple formats.
% Below are superficial introductions of several open and commonly used data models, with focus to the "pure" data content.
% % data, but not the application level.
% As can be seen, only few conceptual data models already involve many schema definition languages and serialization formats.







% ...




