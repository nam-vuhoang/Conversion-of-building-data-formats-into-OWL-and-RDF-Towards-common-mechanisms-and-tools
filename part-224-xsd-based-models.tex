\begin{table*}[h]\footnotesize
    \centering
    \caption{Contents and declaration methods of XML elements}
    \label{tab:xml-elements}
    
    \rowcolors{3}{white}{lightgray}    

    \begin{tabu} to \textwidth { l | c c c | X }
        \hline
            XML element             &   Text    &   Elements    &   Attributes      &   Declaration method     \\
                                    % &   Text    &   Elements    &   Attributes      &                    \\
        \hline
            Simple                  &   Yes     &   No          &   No              &   \texttt{xs:simpleType}                       \\
            Complex empty           &   No      &   No          &   Yes             &   \texttt{xs:complexType} + [\texttt{xs:complexContent}] + attributes \\
            Complex text-only       &   Yes     &   No          &   Yes             &   \texttt{xs:complexType} + \texttt{xs:simpleContent}  + attributes \\
            Complex element-only    &   No      &   Yes         &   Yes             &   \texttt{xs:complexType} + [\texttt{xs:complexContent}] + elements + attributes \\
            Complex mixed           &   Yes     &   Yes         &   Yes             &   \texttt{xs:complexType} + [\texttt{xs:complexContent}] + \newline \texttt{mixed} flag + elements + attributes \\
        \hline
    \end{tabu}
\end{table*}

\subsubsection{XSD-based data models}\label{sec:xsd-based-data-models}

\paragraphx{\\XML elements and XSD types}

Similarly to EXPRESS-based data models, a XSD-based data model is described using one or more interrelated XSD schemas.
Each schema is a kept in a separate document and then the document can be imported by other XSD schemas or XML documents.
For instance, the CityGML data model consists of many XSD schemas, which address specific subdomains such as bridge, building, city furniture, land use, tunnel, etc.
These schemas import other base XSD schemas, such as GML and CityGMLBase.



An XML schema contains declarations of XML elements, their attributes, and related types.
There are two kinds of XML elements: simple and complex.
\emph{Simple elements} can only have text, whereas \emph{complex elements} can contain text, elements, and attributes.
Complex elements also have four subkinds: empty, text-only, elements-only, and mixed.
\autoref{tab:xml-elements} describes possible contents and declaration methods of XML elements to facilitate their recognition.






% \begin{table}\footnotesize
%     \centering
%     \caption{Kinds of XML elements}
%     \label{tab:xml-elements}
    
%     \rowcolors{2}{white}{lightgray}    
    
%     \begin{tabu} to \columnwidth { X c c c }
%         \hline
%             XML element             &   Text    &   Elements    &   Attributes  \\
%         \hline
%             Simple                  &   Yes     &   No          &   No          \\
%             Complex Empty           &   No      &   No          &   Yes         \\
%             Complex Text-only       &   Yes     &   No          &   Yes         \\
%             Complex Element-only    &   No      &   Yes         &   Yes         \\
%             Complex Mixed           &   Yes     &   Yes         &   Yes         \\
%         \hline
%     \end{tabu}
% \end{table}



As can be seen in \autoref{tab:xml-elements}, there are two XSD types.
\emph{Simple types} (\texttt{xs:simpleType}) are used for declaring simple elements or attributes, which only have text content.
\emph{Complex types} (\texttt{xs:complexType}) are applied only for complex elements.

XSD allows to specify a simple type as a list (\texttt{xs:list}) of another simple type, or as a union (\texttt{xs:union}) of other simple types.
Moreover, XSD provides multiple restrictions (\texttt{xs:restriction}) for simple types:
\begin{itemize}
    \item \texttt{enumeration} defines a list of acceptable values.
    \item \texttt{length}, \texttt{min\-Length}, and \texttt{max\-Length} define a number of string characters or list items.
    \item \texttt{min\-Inclu\-sive}, \texttt{max\-Inclu\-sive}, \texttt{min\-Exclu\-sive}, and \texttt{max\-Exclu\-sive} define lower and upper bounds for numeric values.
    \item \texttt{pattern} defines a character pattern for string values.
    \item \texttt{totalDigits} and \texttt{fractionDigits} define numbers of digits in total and after the decimal point allowed in a number.
\end{itemize}



Complex types can include the following declarations: element (\texttt{xs:element}), element group (\texttt{xs:group}), attribute (\texttt{xs:attribute}), attribute group (\texttt{xs:attri\-bute\-Group}), \emph{any}-element (\texttt{xs:any}), and \emph{any}-attribute (\texttt{xs:anyAttribute}).
Any-elements and any-attributes are elements and attributes undefined by the schema.
Elements in a complex type or a element group must be defined with exact one of \emph{order indicators}:
(i) \texttt{sequence} allows child elements to occur in any order;
(ii) \texttt{all} allows child elements to occur in the order as defined;
and (iii) \texttt{choice} allows only one of the child elements to occur.
Regardless of which order indicator is used, elements can be specified with \texttt{min\-Occurs}, \texttt{max\-Occurs} and \texttt{nillable} attributes.
Unlike elements of a complex type, attributes may occur in an element no more than once.
However, attributes can be declared as \texttt{optional} or \texttt{required}.


\bigskip\paragraphx{XSD type hierarchy}







