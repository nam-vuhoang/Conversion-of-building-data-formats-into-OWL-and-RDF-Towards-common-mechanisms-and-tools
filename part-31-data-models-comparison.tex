\subsection{Target criteria}

\textbf{Requirement to input data models}
...


\textbf{Requirement to output data formats}
...

\textbf{Conversion schema}




\subsection{Data models comparison}





% The analysis of the simple and well-defined data models, used in the AEC/FM industry and described in Section \ref{sec:state-of-the-art}, shows that in spite of their different nature, they have a lot of similarities.

In this subsection, the \emph{"pure"} data part of the \emph{simple} and \emph{well-defined} data models, used in the \emph{AEC/FM industry} and described in Section \ref{sec:state-of-the-art}, are compared to each other.
"Pure" data means the static and most important information content needed for data exchange in the industry.
Examples of "dirty" data are: worksheets and text styles in spreadsheet formats, derived attributes, functions and rules in the EXPRESS data schemas.
"Simple" data models do not contain complex elements, such as \texttt{mixed} elements in XSD (see subsection \ref{sec:xsd-based-data-models}), \texttt{BAG}-aggregation data types in EXPRESS (see subsection \ref{sec:express-based-data-models}) or \texttt{image} in relational databases.
"Well-defined" data models are designed by qualified researchers and engineers, using the latest computer science trends and data modeling best practices.
The data models used outside the industry are unpredictable, for instance, their entities may not contain unique keys for linking.
Therefore, data models, which are complicated, or "poorly" designed, or unused in the industry, are outside the scope of the research.
In general, all the mentioned restrictions help to ensure that the data models are \emph{predictable} and probably can be mapped to the dynamic data model, specified in Section 3.2.





In spite of different nature of the EXPRESS-based, the XSD-based, and the relational data models groups, these data models have two significant similarities:

\begin{enumerate}

\item
These data models can be mapped in a certain way to an \emph{object} model. All EXPRESS-based data models are object models by definition. Most of XSD schemas are designed on top of a conceptual model, which involves two basic relationships between different "concepts": \emph{inheritance} and \emph{association}.
The whole-part relationships (aggregations in UML), often used in XSD/XML, are special cases of associations.
These "concepts" are reflected in certain XSD elements and can be "restored" from the XSD to be classes.
In tabular or relational data, each table can be considered as a class, which has associations with other classes.

\item
Most of \emph{single data records} in datasets of these data models can be considered as entity instances.
Examples of single data records are: single lines in SPF, rows in tabular and relational data, or XSD-elements in XML.
However, 


\item
All primitive data types used in the data models are related to the following groups: strings, integers, decimals, logicals, binaries, and date-times.


Entity types are similar to classes in OOP.
Rows in SPF, CSV, spreadsheets and relational databases, or XSD-elements in XML are samples of single data records. 
Rows in SPF are instances of entity types by definition.
\end{enumerate}
First, \emph{most} (not all) of 
In case of tabular and relational data models, the table names can be used to define entity types.
Second, these classes have attributes, but no operations.
Like in EXPRESS or OWL, and unlike traditional OOP, each attribute may have multiple value.




Third, there are only two basic relationships between these classes, namely inheritance and association.
Each class has only one superclass. 


... In the Sections ..., we analyse in details each group of data models, which parts of them can or cannot be converted into OWL and RDF.


% (4) each attribute 

% The main difference between such a class diagram and an object model in traditional object-oriented programming (OOP) languages is that it allows attributes to have multiple values, similarly to properties in OWL languages (see...).



% Firstly, all simple data types are related to one of the following groups: strings, integers, decimals, enumerations, calendar, and logical. 




% The specific features of the data schemas to be ignored, for instance, are: functions, derived attributes and rules in EXPRESS; ... in XSD; data about workbooks, worksheets, text style and so on in Spreadsheets.





